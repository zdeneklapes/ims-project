\documentclass[a4paper, 10pt]{article}
\usepackage[unicode]{hyperref}
\usepackage[utf8x]{inputenc}
\usepackage[IL2]{fontenc}
\usepackage[slovak]{babel}
\usepackage[left=2cm, top=3cm, text={17cm, 24cm}]{geometry}
\usepackage{verbatim}
\usepackage{enumitem}
\usepackage{graphicx}
\usepackage{listings} % vkladanie obrazkov
\usepackage{color}
\usepackage{xcolor}
\usepackage[czech, boxed]{algorithm2e}


\begin{document}
    %%%%%%%%%%%%%%%%%%%%%%%%%%%%%%%% Titulná stránka %%%%%%%%%%%%%%%%%%%%%%%%%%%%%%%%
    \begin{titlepage}
        \begin{center}
            \includegraphics[width=0.77\linewidth]{src/FIT_logo} \\
            \vspace{\stretch{0.382}}
            \Huge{Modelování a simulace -- IMS} \\
            \huge{Okruh 12. -- SHO vo výrobe} \\
            \Large{Dokumentácia}
            \vspace{\stretch{0.618}}
        \end{center}

        \begin{flushleft}
            \Large{Lapeš Zdeněk (xlapes02)} \\
            \Large{Bínovský Andrej (xbinov00)}
        \end{flushleft}
        \vspace{-12mm}
        \hfill\Large{\today}
    \end{titlepage}


    %%%%%%%%%%%%%%%%%%%%%%%%%%%%%%%% Obsah %%%%%%%%%%%%%%%%%%%%%%%%%%%%%%%%
    \tableofcontents
    \newpage

    %%%%%%%%%%%%%%%%%%%%%%%%%%%%%%%% Úvod %%%%%%%%%%%%%%%%%%%%%%%%%%%%%%%%


    \section{Úvod}
    V tejto práci je riešená implementácia procesu výroby chleba, ktorá je použitá na zostavenie modelu testujúceho
    najlepšie možné nakonfigurovanie množstva strojov, pracovníkov a miestností výroby pre rôzne scenáre požiadaviek
    množstva chlebov v daný deň. Danú prácu vypracovali študenti Zdenek Lapeš a Andrej Bínovský z Fakulty informačných
    technológií VUT v Brne.

    \subsection{Čerpanie a konzultácia dát}
    V rámci tejto práce boli dáta použité a konzultované z praxe, ktoré boli získané z výrobného procesu chleba
    z Brnenskej pekárne \texttt{Crocus}. Pre dosianutie čo najvalidnejšieho modelu procesu výroby chleba sme
    spomenutú pekáreň navštívili a získali informácie o problematike každého procesu výroby.

    \subsection{Overovanie validity dát}
    Po celý čas modelovania boli dáta overované a validované na základe komunikácie majitela pekárne. Hned zo
    začiatku sme si namodelovali a porovnali presnú konfiguráciu spomenutej pekárne. Čo nám docielilo overenie
    presnosti modelu na základe porovnania skutočného času výroby s časom výroby chleba v modeli.


    %%%%%%%%%%%%%%%%%%%%%%%%%%%%%%%% Rozbor témy a použitých metód/technológií %%%%%%%%%%%%%%%%%%%%%%%%%%%%%%%%
    \section {Rozbor témy}\label{sec:rozbor-temy}
    Postup výroby chleba sa skladá z viacerych procesov, ktoré sú navzájom závislé:
    \begin{itemize}
        \item \textbf{Výroba cesta} -- Proces výroby cesta spočíva zo zmiešania všetkých surovín do jedného. Počet
        a dostupnosť surovín sa v modeli neberie k úvahe. Jeden proces výroby cesta reprezentuje jeden mixér a
        výsledokm je \textbf{140kg cesta}. Proces trvá \textbf{10 minút} \textpm \textbf{ 2 minúty}.

        \item \textbf{Krájanie cesta na bochníky} -- Krájanie cesta na bochníky je proces, ktorý sa vykonáva pracovníkom
        ručne. V modeli teda platí, že jeden pracovník sa rovná jednému stolu na krájanie. Vstupom procesu je
        \textbf{1kg cesta} a výstupom je \textbf{1 bochník}. Proces trvá \textbf{30 sekúnd} \textpm \textbf{ 5 sekúnd}.

        \item \textbf{Fermentácia bochníkov} -- Po nakrájaní sa bochníky ukladajú na plech do vozíka.
        Jeden vozík obsahuje miesto na \textbf{70 bochníkov}. Po naplnení sa vozík odvezie do fermentačnej miestnosti
        na čas  \textbf{20 minút} \textpm \textbf{ 2 minúty}.

        \item \textbf{Pečenie chleba} -- Po fermentácii sa bochníky na vozíkoch uložia do pece, kde sa pečú.
        Jedna pec má kapacitu \textbf{1 vozíku}, teda \textbf{70 bochníkov}. Proces trvá \textbf{30 minút} \textpm \textbf{ 2 minúty}.

        \item \textbf{Balenie chleba} -- Po upečení pracovníci roztrieda chleba do bedien. Balenie jedného chleba
        praconíkom trvá \textbf{10 sekúnd} \textpm \textbf{ 3 sekundy}.
    \end{itemize}

    \subsection{Použité metódy, postupy a technológie}
    Na implementáciu modelu bol použitý programovací jazyk \texttt{C++} za podpory simulačnej knihonvne \texttt{SimLib}.
    [TODO: doplnit odkaz na knihovnu] [TODO: zdůvodnění, proč jsou pro zadaný problém vhodné. Zdůvodnění může být podpořeno ukázáním alternativního přístupu a srovnáním s tím vaším]
    \subsection{Pôvod metód a technológií}
    [TODO]

    %%%%%%%%%%%%%%%%%%%%%%%%%%%%%%%% Implementácia programu %%%%%%%%%%%%%%%%%%%%%%%%%%%%%%%%
    \section {Koncepcia modelu}
    Konceptuálny model pekárne je zobrazený pomocou petriho siete\ref{fig:conceptual_model}.
    V modeli je zobrazený zjednodušený proces výroby chleba. Napriek zjednodušeniu modelu je zachovaná korektnosť simulácie.
    \begin{itemize}
        \item \textbf{Výroba cesta} -- Pri prvom procese je zanedbateľná dostupnosť a naskladnovaie surovín,
        pretože to zaobstaráva externá firma. Výstup a čas procesu je totožný s realitou.

        \item \textbf{Krájanie cesta na bochníky} -- Pri druhom procese sme reálne dáta dekomponovali. Získané
        dáta z pekárne reprezentovali čas nakrájania 70 bochníkov (zo 70 kg cesta) na jeden vozík pracovníkmi za čas \textbf{10 min} \textpm \textbf{ 2 minúty}.
        Pre systém sme zvolili proces krájania 1kg cesta na 1 bochník, 1 pracovníkom za čas \textbf{30 sekúnd} \textpm \textbf{ 5 sekúnd}.

        \item \textbf{Fermentácia bochníkov} -- Pri procese fermentácie sme zachovali vstupy, výstupy a časy na základe reálnych dát.

        \item \textbf{Pečenie chleba} -- Pri procese pečenia sme zachovali vstupy, výstupy a časy na základe reálnych dát.

        \item \textbf{Balenie chleba} -- Proces balanie chleba reprezentuje viacero podprocesov ako napríklad príprava a čistenie bedien.
        Balenie spočíva v roztriedení chleba do bedien a ich následne uloženie do skladu. Reálny proces bol meraný na
        zabalenie jedného vozíku 70 chlebov do bedien za čas \textbf{10 min} \textpm \textbf{ 2 minúty}. Jedna bedna môže obsahovať 3 až 4 chleby. Proces v modeli sme dekomponovali
        na zabalenie jedného chleba jedným pracovníkom za čas \textbf{10 sekúnd} \textpm \textbf{ 3 sekundy}.
    \end{itemize}

    \section{Architektúra simulačného modelu}
    Simulacni model\cite[slide 44]{IMS_slides} lze spustit ve 3 rezimech. Nastaveni techto
    rezimu je urceno konstantami v souboru \texttt{src/macros.h}.

    Kazdy proces ma nasledujici implementaci funkce \texttt{Behavior()}:

    Na zacatku se spusti vsechny mixer process, popriapde se vytvroi fronta na mixer process.
    \begin{algorithm}[ht]
        \SetKw{Break}{break}

        Zahajeni order process\;
        \While{nemixuje se testo pro upeceni vsech chlebu}
        {
            Zahajeni mixer process\;
        }
        čekání, až budou vsechny chleby v bednickach\;
        ukončeni Order process\;

        \caption{Zahajeni procesu výroby chleba}
        \label{algorithm:orderprocess}
    \end{algorithm}

    \begin{algorithm}[ht]
        \SetKw{Break}{break}
        Cekej dokud neni dany process hotovy\;
        Spust nasledujici process\;
        \caption{Process výroby chleba}
        \label{algorithm:otherprocess}
    \end{algorithm}

    \subsection{Vstupy modelu}
    Vstupem simulacniho modelu jsou atributy tridy \texttt{Args} v souboru \texttt{src/Args.h}:
    \begin{itemize}
        \item breads -- pocet chlebov, ktere se maji vyrobit
        \item mixers -- pocet mixeru
        \item tables -- pocet stolu na krájeni
        \item carts -- pocet voziku
        \item ovens -- pocet peci
        \item packers -- pocet balicich pracovniku
    \end{itemize}

    Dalsi atributy tridy \texttt{Args} jsou pro ovladani simulace a vystupniho souboru:
    \begin{itemize}
        \item outfile -- nazev vystupniho souboru
        \item simulations -- pocet simulaci, ktere se maji spustit
    \end{itemize}

    Tyto vstupy jsou zadavany pri spusteni programu, viz.\ref{subsec:spousteni-simulacniho-model}

    \subsection{Výstup simulácie}
    Výstupem simulácie je mozen ridit pres konstanty v souboru \texttt{src/macros.h}:
    \begin{itemize}
        \item SIMULACNI REZIM (DEBUG=0 a TEST=0) -- v tomto rezimu je mozne spustit simulaci a sledovat vysledky, model se spusti 3x.
        \item DEBUG REZIM (DEBUG=1) -- v tomto rezimu je mozne sledovat podrobnejsi info o probihajici simulaci
        v jednotlivych krocich (doba vyrizeni jednotlivych procesu, vstupy a vystupy jednotlivych procesu, \ldots)
        \item EXPERIMENT REZIM (TEST=1) -- v tomto rezimu je spusteno experimentovani pro zjisteni,
        ktera konfigurace modelu (vstupu), jse nejvhodnejsi pro peceni
        chleba z hlediska nejmensiho casu vyrizeni tzv. optimalni konfigurace.
    \end{itemize}

    Výstupom simulácie jsou data o čase výroby chleba, zatizeni jednotlivich stroju a pracovniku.
    Vystup muze byt ulozen do souboru, ktery je zadany prikazem \texttt{--outfile}, pokud neni zadan
    vystup je vytisten na standardni vystup.

    \subsection{Spousteni simulacniho model}\label{subsec:spousteni-simulacniho-model}
    Simulacni model se musi pred spustenim kompilovat pomoci Makefile.
    \texttt{make} a spustim model je mozne pomoci prikazu \texttt{make run} s defaultnimi hodnotami vstupu.
    Model lze spoustet i s parametry, ktere jsou vstupy simulacniho modelu, viz. \texttt{src/Args.h}:
    Spousteni: \texttt{./build/bread\_factory --breads 100 --mixers 2 --tables 2
    --fermentations 2 --ovens 3 --loads 3 --simulations 3 --outfile out.txt}.


    \section{Podstata imulačných experimentov a ich priebeh}


    \section{Zhrnutie simulašnych experimentov}

    \section {Čerpanie zdrojov}
    {\cite{example}}

    \newpage
    \bibliographystyle{czechiso}
    \bibliography{documentation}
\end{document}
