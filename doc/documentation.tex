\documentclass[a4paper, 10pt]{article}
\usepackage[unicode]{hyperref}
\usepackage[utf8x]{inputenc}
\usepackage[IL2]{fontenc}
\usepackage[slovak]{babel}
\usepackage[left=2cm, top=3cm, text={17cm, 24cm}]{geometry}
\usepackage{verbatim}
\usepackage{enumitem}
\usepackage{graphicx}
\usepackage{listings} % vkladanie obrazkov
\usepackage{color}
\usepackage{xcolor}
\usepackage[czech, boxed]{algorithm2e}


\begin{document}
%%%%%%%%%%%%%%%%%%%%%%%%%%%%%%%% Titulná stránka %%%%%%%%%%%%%%%%%%%%%%%%%%%%%%%%
    \begin{titlepage}
        \begin{center}
            \includegraphics[width=0.77\linewidth]{src/FIT_logo} \\
            \vspace{\stretch{0.382}}
            \Huge{Modelování a simulace -- IMS} \\
            \huge{Okruh 12. -- SHO vo výrobe} \\
            \Large{Dokumentácia}
            \vspace{\stretch{0.618}}
        \end{center}

        \begin{flushleft}
            \Large{Lapeš Zdeněk (xlapes02)} \\
            \Large{Bínovský Andrej (xbinov00)}
        \end{flushleft}
        \vspace{-12mm}
        \hfill\Large{\today}
    \end{titlepage}


%%%%%%%%%%%%%%%%%%%%%%%%%%%%%%%% Obsah %%%%%%%%%%%%%%%%%%%%%%%%%%%%%%%%
    \tableofcontents
    \newpage

%%%%%%%%%%%%%%%%%%%%%%%%%%%%%%%% Úvod %%%%%%%%%%%%%%%%%%%%%%%%%%%%%%%%


    \section{Úvod}
    V tejto práci je riešená implementácia procesu výroby chleba, ktorá je použitá na zostavenie modelu testujúceho
    najlepšie možné nakonfigurovanie množstva strojov, pracovníkov a miestností výroby pre rôzne scenáre požiadaviek
    množstva chlebov v daný deň. Danú prácu vypracovali študenti Zdenek Lapeš a Andrej Bínovský z Fakulty informačných
    technológií VUT v Brne.

    \subsection{Čerpanie a konzultácia dát}
    V rámci tejto práce boli dáta použité a konzultované z praxe, ktoré boli získané z výrobného procesu chleba
    z Brnenskej pekárne \texttt{Crocus}\cite{Crocus}. Pre dosiahnutie čo najvalidnejšieho modelu procesu výroby chleba sme
    spomínanú pekáreň navštívili a získali informácie o problematike každého procesu výroby.

    \subsection{Overovanie validity dát}
    Po celý čas modelovania boli dáta overované a validované na základe komunikácie majiteľa pekárne. Hned zo
    začiatku sme si namodelovali a porovnali presnú konfiguráciu spomenutej pekárne. Čo nám docielilo overenie
    presnosti modelu na základe porovnania skutočného času výroby s časom výroby chleba v modeli.


%%%%%%%%%%%%%%%%%%%%%%%%%%%%%%%% Rozbor témy a použitých metód/technológií %%%%%%%%%%%%%%%%%%%%%%%%%%%%%%%%
    \section {Rozbor témy}\label{sec:rozbor-temy}
    Postup výroby chleba sa skladá z viacerých procesov, ktoré sú navzájom závislé:
    \begin{itemize}
        \item \textbf{Výroba cesta} -- Proces výroby cesta spočíva zo zmiešania všetkých surovín do jedného. Počet
        a dostupnosť surovín sa v modeli neberie k úvahe. Jeden proces výroby cesta reprezentuje jeden mixér a
        výsledkom je \textbf{140 kg cesta}. Proces trvá \textbf{10 minút} \textpm \textbf{ 2 minúty}.

        \item \textbf{Krájanie cesta na bochníky} -- Krájanie cesta na bochníky je proces, ktorý sa vykonáva pracovníkom
        ručne. V modeli teda platí, že jeden pracovník sa rovná jednému stolu na krájanie. Vstupom procesu je
        \textbf{1kg cesta} a výstupom je \textbf{1 bochník}. Proces trvá \textbf{30 sekúnd} \textpm \textbf{ 5 sekúnd}.

        \item \textbf{Fermentácia bochníkov} -- Po nakrájaní sa bochníky ukladajú na plech do vozíka.
        Jeden vozík obsahuje miesto na \textbf{70 bochníkov}. Po naplnení sa vozík odvezie do fermentačnej miestnosti
        na čas  \textbf{20 minút} \textpm \textbf{ 2 minúty}.

        \item \textbf{Pečenie chleba} -- Po fermentácii sa bochníky na vozíkoch uložia do pece, kde sa pečú.
        Jedna pec má kapacitu \textbf{1 vozíku}, teda \textbf{70 bochníkov}. Proces trvá \textbf{30 minút} \textpm \textbf{ 2 minúty}.

        \item \textbf{Balenie chleba} -- Po upečení pracovníci roztriedia chleba do bedien. Balenie jedného chleba
        pracovníkom trvá \textbf{10 sekúnd} \textpm \textbf{ 3 sekundy}.
    \end{itemize}

    \subsection{Použité metódy, postupy a technológie}
    Na implementáciu modelu bol použitý programovací jazyk \texttt{C++} za podpory simulačnej knižnice \texttt{SimLib}.
    Hlavným dôvodom použitia jazyka \texttt{C++} je využitie naimplementovanej
    knižnice \texttt{SimLib}. Knižnicu sme si vybrali z dôvodu modelovania procesov výroby na základe petriho siete, ktorá
    toto modelovanie podporuje. Príklady a správne použitie knižnice sme čerpali z prednášok predmetu \texttt{IMS}\cite{IMS_slides}.
    Na kompiláciu sme využili nástroj \texttt{CMake}. Na vývoj bol zvolený operačný systém
    \texttt{Linux Ubuntu}, ktorý bežal vo virtualizovanom prostredí nástroja \texttt{Docker}.

    \subsection{Pôvod metód a technológií}
    \begin{itemize}
        \item \textbf{C++} -- Verzia \texttt{C++20}: \texttt{https://en.cppreference.com/w/cpp/20}.
        \item \textbf{SimLib} -- Verzia \texttt{3.09}\cite{simblib}. Autori knižnice sú Petr Peringer,
        David Leska a David Martinek.
        \item \textbf{CMAKE} -- Verzia \texttt{3.5}: \texttt{https://cmake.org/}.
        \item \textbf{Docker} -- Verzia \texttt{20.10.17}: \texttt{https://www.docker.com/}.
        \item \textbf{Linux Ubuntu} -- Verzia \texttt{20.04}: \texttt{https://ubuntu.com/}.
    \end{itemize}


%%%%%%%%%%%%%%%%%%%%%%%%%%%%%%%% Implementácia programu %%%%%%%%%%%%%%%%%%%%%%%%%%%%%%%%
    \section {Koncepcia modelu}
    Konceptuálny model pekárne je zobrazený pomocou petriho siete \ref{fig:petri}.
    V modeli je zobrazený zjednodušený proces výroby chleba. Napriek zjednodušeniu modelu je zachovaná korektnosť simulácie.
    \begin{itemize}
        \item \textbf{Výroba cesta} -- Pri prvom procese je zanedbateľná dostupnosť a naskladnovanie surovín,
        pretože to zaobstaráva externá firma. Výstup a čas procesu je totožný s realitou.

        \item \textbf{Krájanie cesta na bochníky} -- Pri druhom procese sme reálne dáta dekomponovali. Získané
        dáta z pekárne reprezentovali čas nakrájania 70 bochníkov (zo 70 kg cesta) na jeden vozík pracovníkmi za čas \textbf{10 min} \textpm \textbf{ 2 minúty}.
        Pre systém sme zvolili proces krájania 1kg cesta na 1 bochník, 1 pracovníkom za čas \textbf{30 sekúnd} \textpm \textbf{ 5 sekúnd}.

        \item \textbf{Fermentácia bochníkov} -- Pri procese fermentácie sme zachovali vstupy, výstupy a časy na základe reálnych dát.

        \item \textbf{Pečenie chleba} -- Pri procese pečenia sme zachovali vstupy, výstupy a časy na základe reálnych dát.

        \item \textbf{Balenie chleba} -- Proces balenie chleba reprezentuje viacero podprocesov ako napríklad príprava a čistenie bedien.
        Balenie spočíva v roztriedení chleba do bedien a ich následne uloženie do skladu. Reálny proces bol meraný na
        zabalenie jedného vozíku 70 chlebov do bedien za čas \textbf{10 min} \textpm \textbf{ 2 minúty}. Jedna bedna môže obsahovať 3 až 4 chleby. Proces v modeli sme dekomponovali
        na zabalenie jedného chleba jedným pracovníkom za čas \textbf{10 sekúnd} \textpm \textbf{ 3 sekundy}.
    \end{itemize}

    \begin{figure}[h]
        \center{\fbox{\scalebox{0.3}{\includegraphics{src/petri}}}}
        \caption{\label{fig:petri}Petriho sieť modelu pekárne}
    \end{figure}


    \section{Architektúra simulačného modelu}
    Simulačný model\cite[slide 44]{IMS_slides} je možné spustiť v 3 režimoch. Nastavenie týchto
    režimov je určené konštantami v súbore \texttt{src/macros.h}.
    Každý proces má nasledujúcu implementáciu funkcie \texttt{Behavior()}:
    Na začiatku sa spustia všetky \texttt{mixer process}, poprípade se vytvori fronta na \texttt{mixer process}.

    \begin{algorithm}[ht]
        \SetKw{Break}{break}

        Zahájenie order process\;
        \While{nemixuje sa cesto pre upečenie všetkých chlebov}
        {
            Zahájenie mixer process\;
        }
        čakanie, až budú všetky chleby v bedničkách\;
        ukončenie Order process\;

        \caption{Zahájenie procesu výroby chleba}
        \label{algorithm:orderprocess}
    \end{algorithm}

    \begin{algorithm}[ht]
        \SetKw{Break}{break}
        Čakaj pokiaľ nie je daný proces hotový\;
        Spusť nasledujúci proces\;
        \caption{Proces výroby chleba}
        \label{algorithm:otherprocess}
    \end{algorithm}

    \subsection{Vstupy modelu}
    Vstupom simulačného modelu sú atribúty triedy \texttt{Args} v súbore \texttt{src/Args.h}:
    \begin{itemize}
        \item breads -- počet chlebov, ktoré se majú vyrobiť
        \item mixers -- počet mixérov
        \item tables -- počet stolov na krájanie
        \item carts -- počet vozíkov
        \item ovens -- počet pecí
        \item packers -- počet pracovníkov na balenie
    \end{itemize}

    Ďalšie atribúty triedy \texttt{Args} sú pre ovládanie simulácie a výstupného súboru:
    \begin{itemize}
        \item outfile -- názov výstupného súboru
        \item simulations -- počet simulácií, ktoré sa majú spustiť
    \end{itemize}

    Tieto vstupy sú zadané pri spustení programu, viz.\ref{subsec:spousteni-simulacniho-model}

    \subsection{Výstup simulácie}
    Výstupom simulácie je možné riadiť cez konštanty v súbore \texttt{src/macros.h}:
    \begin{itemize}
        \item SIMULAČNÝ REŽIM (DEBUG=0 a TEST=0) -- v tomto režimu je možné spustiť simuláciu a sledovať výsledky, model se spustí 3x.
        \item DEBUG REŽIM (DEBUG=1) -- v tomto režimu je možné sledovať podrobnejšie informácie o prebiehajúcich simuláciách
        v jednotlivých krokoch (doba vybavenia jednotlivých procesov, vstupy a výstupy jednotlivých procesov, \ldots)
        \item EXPERIMENT REŽIM (TEST=1) -- v tomto režimu je spustené experimentovanie pre zistenie,
        ktorá konfigurácia modelu (vstupu), je najvhodnejšia pre pečenie
        chleba z hľadiska najmenšieho času vybavenia tzv. optimalnej konfigurácie.
    \end{itemize}

    Výstupom simulácie sú dáta o čase výroby chleba, zaťaženie jednotlivých strojov a pracovníkov.
    Výstup môže byť uložený do súboru, ktorý je zadaný príkazom \texttt{--outfile}, keď nie je zadaný
    výstup je vytlačený na štandardný výstup.

    \subsection{Spúšťanie simulačného modelu}\label{subsec:spousteni-simulacniho-model}
    Simulačný model se musí pred spustením skompilovať pomocou \texttt{Makefile}.
    \texttt{make} a spustený model je možné pomocou príkazu \texttt{make run} s predvolenými hodnotami vstupu.
    Model je možné spúšťať aj s parametrami, ktoré sú vstupmi simulačného modelu, viz. \texttt{src/Args.h}:
    Spúšťanie: \texttt{./build/bread\_factory --breads 100 --mixers 2 --tables 2
    --fermentations 2 --ovens 3 --loads 3 --simulations 3 --outfile out.txt}.


    \section{Podstata simulačných experimentov a ich priebeh}
    Cieľom práce, ktorá má simulovať výrobu chleba je zistiť, či pekáreň s ktorou pracujeme má najlepšiu konfiguráciu
    strojov, miestností a pracovníkov pre výrobu chleba.

    \subsection{Postup experimentovania}
    Pre experimentovanie je vytvorená funkcia \texttt{experiments()}, ktorá po spustení prebehne všetky možné
    konfigurácie počtu prostriedkov v rozmedzí 1 až 9.

    \subsection{Priebeh experimentov}
    Pri modelovaní výroby sme rozdelili prostriedky na konštanty a premenné.\\ \\
    \textbf{Konštanty:}
    \begin{itemize}
        \item Výstup jedného procesu mixovania -- \texttt{140 kg cesta}
        \item Maximálna kapacita bochníkov vo vozíku -- \texttt{70 kusov}
        \item Maximálna kapacita vozíkov v peci -- \texttt{1 vozík}
    \end{itemize}
    \textbf{Premenné:}
    \begin{itemize}
        \item Počet chlebov na výrobu
        \item Počet mixérov
        \item Počet stolov/pracovníkov na krájanie
        \item Maximálna kapacita vozíkov vo fermentačnej miestnosti
        \item Počet pecí
        \item Počet pracovníkov na balenie
    \end{itemize}
    \textbf{Výsledky experimentov:}\\

    \subsubsection{Validácia modelu}
    \textbf{Reálna konfigurácia pekárne (simulácia)}
    \begin{center}
        \begin{tabular}{ |p{2cm}|p{2cm}|p{2cm}|p{2cm}|p{2cm}|p{2cm}|p{2cm}| }
            \hline
            čas / min & chleby / ks & mixéry & pracovníci na krájanie & kapacita fermentačnej miestnosti & kapacita pecí & pracovníci na balenie \\
            \hline\hline
            147       & 140         & 2      & 2                      & 8                                & 2             & 2                     \\ \hline
            209       & 280         & 2      & 2                      & 8                                & 2             & 2                     \\ \hline
            288       & 420         & 2      & 2                      & 8                                & 2             & 2                     \\ \hline
        \end{tabular}
    \end{center}

    \textbf{Reálny konfigurácia pekárne (reálny získaný čas)}
    \begin{center}
        \begin{tabular}{ |p{2cm}|p{2cm}|p{2cm}|p{2cm}|p{2cm}|p{2cm}|p{2cm}| }
            \hline
            čas / min & chleby / ks & mixéry & pracovníci na krájanie & kapacita fermentačnej miestnosti & kapacita pecí & pracovníci na balenie \\
            \hline\hline
            150       & 140         & 2      & 2                      & 8                                & 2             & 2                     \\ \hline
        \end{tabular}
    \end{center}

    Ako môžeme vidieť reálny čas so simulovaným časom výroby chleba je totožný. Vďaka tomuto experimentu
    môžeme pokračovať v experimentovaní s inými konfiguráciami, ktoré nám ukážu efektivitu pekárne.

    \subsubsection{Najlepšia konfigurácia experimentov (simulácia)}
    \begin{center}
        \begin{tabular}{ |p{2cm}|p{2cm}|p{2cm}|p{2cm}|p{2cm}|p{2cm}|p{2cm}| }
            \hline
            čas / min & chleby / ks & mixéry & pracovníci na krájanie & kapacita fermentačnej miestnosti & kapacita pecí & pracovníci na balenie \\
            \hline\hline
            112       & 140         & 2      & 4                      & 6                                & 8             & 6                     \\ \hline
            125       & 280         & 3      & 4                      & 9                                & 5             & 9                     \\ \hline
            131       & 420         & 4      & 6                      & 6                                & 8             & 9                     \\ \hline
        \end{tabular}
    \end{center}

    Z tabulky vypliva, ze mixeru je mozne
    mit v pekarni najmenej, nebot maji velikou produkovatelnost testa a zaroven
    tento proces netrva prilis dlouho.

    Kapacita fermentacni mistnosti a kapacita peci
    je vhodne drzet v pomeru $2:3$ s prihlednutim ke kapacite peci, nebot
    peceni trva o \textbf{10 minut $\pm$ 2 minuty} dele, nez fermentace.

    Vice pracovniku je dobre mit k dispozici prave v dobe krajeni a baleni, protoze
    je to proces, ktery je nejvice zavisly na lidskem faktoru.

    \subsubsection{Najhoršia konfigurácia experimentov (simulácia):}
    \begin{center}
        \begin{tabular}{ |p{2cm}|p{2cm}|p{2cm}|p{2cm}|p{2cm}|p{2cm}|p{2cm}| }
            \hline
            čas / min & chleby / ks & mixéry & pracovníci na krájanie & kapacita fermentačnej miestnosti & kapacita pecí & pracovníci na balenie \\
            \hline\hline
            228       & 140         & 3      & 1                      & 3                                & 6             & 9                     \\ \hline
            327       & 280         & 6      & 1                      & 2                                & 7             & 2                     \\ \hline
            419       & 420         & 8      & 1                      & 4                                & 1             & 2                     \\ \hline
        \end{tabular}
    \end{center}

    Z tabulky vypliva, ze pokud je k dispozici malo pracovniku pri baleni a krajeni,
    tak se celkovy cas vyrizeni objednavky znacne zvysi.

    Rovnez vysledky dokazuji, ze je lepsi mit vetsi kapacitu fermentacni mistnosti
    a kapcitu peci nez mixeru.

    \subsection{Závery experimentov}
    Pri experimentovaní bolo spustených 177 147 konfigurácií, z ktorých sme zistili najlepšie a najhoršie nakonfigurovanie
    pekárne pre výrobu 140, 280 a 420 chlebov. Týmito výsledkami vieme tvrdit že pekáreň nemá najefektívnejšie nakonfigurovanie.
    Najlepší výsledok pre 140 chlebov bol o \textbf{38 minút} rýchlejší ako pri reálnom čase. Z priložených tabuliek
    môžme vidieť aj konfigurácie pre prípad výroby 280 a 420 chlebov. Do výsledkov sme taktiež zahrnuli najhoršie nakonfigurovanie
    pre prípad vyvarovania sa týchto konfigurácií.


    \section{Záver}
    Ako bolo uvedené v predchádzajúcej kapitole, bolo spustených 177 147 konfigurácií. Týmito výsledkami vieme tvrdiť
    že pekáreň nie je najefektívnejšia.
    Tieto dáta poskytneme pekárni, ktorá môže tieto výsledky využiť pre zefektívnenie procesu výroby.
    Musíme taktiež ale povedať, že táto konfigurácia nemusí byť najvhodnejšia vzhľadom na cenu energií, prostriedkov a
    platov za pracovníkov. Týmto chceme upozorniť čitateľa, že model bol navrhnutý pre dosiahnutie najrýchlejšieho času
    procesu výroby a nebol zohľadnený faktor financovania. Pre ďalší rozvoj práce by bolo vhodné zohľadniť aj tento
    dôležitý faktor.


    \newpage
    \bibliographystyle{czechiso}
    \bibliography{documentation}
\end{document}
