\documentclass[a4paper, 10pt]{article}
\usepackage[unicode]{hyperref}
\usepackage[utf8x]{inputenc}
\usepackage[IL2]{fontenc}
\usepackage[slovak]{babel}
\usepackage[left=2cm, top=3cm, text={17cm, 24cm}]{geometry}
\usepackage{verbatim}
\usepackage{enumitem}
\usepackage{graphicx}
\usepackage{listings} % vkladanie obrazkov
\usepackage{color}
\usepackage{xcolor}


\begin{document}
    %%%%%%%%%%%%%%%%%%%%%%%%%%%%%%%% Titulná stránka %%%%%%%%%%%%%%%%%%%%%%%%%%%%%%%%
    \begin{titlepage}
        \begin{center}
            \includegraphics[width=0.77\linewidth]{src/FIT_logo} \\
            \vspace{\stretch{0.382}}
            \Huge{Modelování a simulace -- IMS} \\
            \huge{Okruh 12. -- SHO vo výrobe} \\
            \Large{Dokumentácia}
            \vspace{\stretch{0.618}}
        \end{center}

        \begin{flushleft}
            \Large{Lapeš Zdeněk (xlapes02)} \\
            \Large{Bínovský Andrej (xbinov00)}
        \end{flushleft}
        \vspace{-12mm}
        \hfill\Large{\today}
    \end{titlepage}


    %%%%%%%%%%%%%%%%%%%%%%%%%%%%%%%% Obsah %%%%%%%%%%%%%%%%%%%%%%%%%%%%%%%%
    \tableofcontents
    \newpage

    %%%%%%%%%%%%%%%%%%%%%%%%%%%%%%%% Úvod %%%%%%%%%%%%%%%%%%%%%%%%%%%%%%%%
    \section{Úvod}
        V tejto práci je riešená implementácia procesu výroby chleba, ktorá je použitá na zostavenie modelu testujúceho
        najlepšie možné nakonfigurovanie množstva strojov, pracovníkov a miestností výroby pre rôzne scenáre požiadaviek
        množstva chlebov v daný deň. Danú prácu vypracovali študenti Zdenek Lapeš a Andrej Bínovský z Fakulty informačných
        technológií VUT v Brne.
        \subsection{Čerpanie a konzultácia dát}
            V rámci tejto práce boli dáta použité a konzultované z praxe, ktoré boli získané z výrobného procesu chleba
            z Brnenskej pekárne \texttt{Crocus}. Pre dosianutie čo najvalidnejšieho modelu procesu výroby chleba sme
            spomenutú pekáreň navštívili a získali informácie o problematike každého procesu výroby.
        \subsection{Overovanie validity dát}
            Po celý čas modelovania boli dáta overované a validované na základe komunikácie majitela pekárne. Hned zo
            začiatku sme si namodelovali a porovnali presnú konfiguráciu spomenutej pekárne. Čo nám docielilo overenie
            presnosti modelu na základe porovnania skutočného času výroby s časom výroby chleba v modeli.


    %%%%%%%%%%%%%%%%%%%%%%%%%%%%%%%% Rozbor témy a použitých metód/technológií %%%%%%%%%%%%%%%%%%%%%%%%%%%%%%%%
    \section {Rozbor témy}
            Postup výroby chleba sa skladá z viacerych procesov, ktoré sú navzájom závislé:
            \begin{itemize}
                \item \textbf{Výroba cesta} -- Proces výroby cesta spočíva zo zmiešania všetkých surovín do jedného. Počet
                a dostupnosť surovín sa v modeli neberie k úvahe. Jeden proces výroby cesta reprezentuje jeden mixér a
                výsledokm je \textbf{140kg cesta}. Proces trvá \textbf{10 minút}.

                \item \textbf{Krájanie cesta na bochníky} -- Krájanie cesta na bochníky je proces, ktorý sa vykonáva pracovníkom
                ručne. V modeli teda platí, že jeden pracovník sa rovná jednému stolu na krájanie. Vstupom procesu je
                \textbf{1kg cesta} a výstupom je \textbf{1 bochník}. Proces trvá \textbf{0.5 minúty}.

                \item \textbf{Fermentácia bochníkov} -- Po nakrájaní sa bochníky ukladajú na plech do vozíka.
                Jeden vozík obsahuje miesto na 70 bochníkov. Po naplnení sa vozík odvezie do fermentačnej miestnosti
                na čas \textbf{TODO}.

                \item \textbf{Pečenie chleba} -- Po fermentácii sa bochníky na vozíkoch uložia do pece, kde sa pečú.
                Jedna pec má kapacitu 1 vozíku, teda 70 bochníkov. Proces trvá \textbf{TODO}.

                \item \textbf{Balenie chleba} -- Po upečení pracovníci roztrieda chleba do bedien. Balenie vykonané jedným
                praconíkom trvá \textbf{TODO}. Výsledkom je spracovanie \textbf{TODO} chlebov.
            \end{itemize}
        \subsection{Použité metódy, postupy a technológie}

        \subsection{Pôvod metód a technológií}


    %%%%%%%%%%%%%%%%%%%%%%%%%%%%%%%% Implementácia programu %%%%%%%%%%%%%%%%%%%%%%%%%%%%%%%%
    \section {Koncepcia modelu}


    \section{Architektúra simulačného modelu}

    \section{Podstata imulačných experimentov a ich priebeh}

    \section{Zhrnutie simulašnych experimentov}

    \section {Čerpanie zdrojov}
        {\cite{example}}

    \newpage
    \bibliographystyle{czechiso}
    \bibliography{documentation}
\end{document}
